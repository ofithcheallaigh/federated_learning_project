\documentclass[conference]{IEEEtran}
\IEEEoverridecommandlockouts
% The preceding line is only needed to identify funding in the first footnote. If that is unneeded, please comment it out.
\usepackage{cite}
\usepackage{amsmath,amssymb,amsfonts}
\usepackage{algorithmic}
\usepackage{graphicx}
\usepackage{textcomp}
\usepackage{xcolor}
\usepackage{parskip}
\def\BibTeX{{\rm B\kern-.05em{\sc i\kern-.025em b}\kern-.08em
    T\kern-.1667em\lower.7ex\hbox{E}\kern-.125emX}}
\begin{document}

\title{Research Notes\\
{\footnotesize \textsuperscript{*}Note: Sub-titles are not captured in Xplore and
should not be used}
\thanks{Identify applicable funding agency here. If none, delete this.}
}


\author{
\IEEEauthorblockN{Seán Ó Fithcheallaigh}
\IEEEauthorblockA{\textit{Department of Computing} \\
\textit{Ulster University}\\
Belfast, N. Ireland \\
sofithcheallaigh@gmail.com or o\_fithcheallaigh-s@ulster.ac.uk}
}

\maketitle

\begin{abstract}
These notes are taken from several research, which will be listed in the reference section. These notes are written with a view to understanding a potential research area for a PhD. \\
\noindent These notes are for personal use.
\end{abstract}

\begin{IEEEkeywords}

\end{IEEEkeywords}

\section{Notes}
This section of the paper will contain the notes from each paper. Each paper will be detailed in turn.

\subsection{Enabling and Leveraging AI in the Intelligent Edge:
A Review of Current Trends and Future Directions}

The first paper reviewed is on enabling and leveraging AI in the Intelligent Edge \cite{b1}. In the \textbf{Introduction}, the authors start by discussing why computational tasks and processes have been moved to fog networks, or to the network edge. They discuss discuss how this migration is a necessary step in the emergence of various "Smart" applications, and eventually smart cities. The idea is in these smart cities, AI would be deployed as when it was needed, and where it was needed. 


\section*{References}

\begin{thebibliography}{00}
\bibitem{b1} T. Goethals, B. Volckaert and F. D. Turck, "Enabling and Leveraging AI in the Intelligent Edge: A Review of Current Trends and Future Directions," in IEEE Open Journal of the Communications Society, vol. 2, pp. 2311-2341, 2021, doi: 10.1109/OJCOMS.2021.3116437.

% \bibitem{b1} G. Eason, B. Noble, and I. N. Sneddon, ``On certain integrals of Lipschitz-Hankel type involving products of Bessel functions,'' Phil. Trans. Roy. Soc. London, vol. A247, pp. 529--551, April 1955.
% \bibitem{b2} J. Clerk Maxwell, A Treatise on Electricity and Magnetism, 3rd ed., vol. 2. Oxford: Clarendon, 1892, pp.68--73.
% \bibitem{b3} I. S. Jacobs and C. P. Bean, ``Fine particles, thin films and exchange anisotropy,'' in Magnetism, vol. III, G. T. Rado and H. Suhl, Eds. New York: Academic, 1963, pp. 271--350.
% \bibitem{b4} K. Elissa, ``Title of paper if known,'' unpublished.
 %\bibitem{b5} R. Nicole, ``Title of paper with only first word capitalized,'' J. Name Stand. Abbrev., in press.
 %\bibitem{b6} Y. Yorozu, M. Hirano, K. Oka, and Y. Tagawa, ``Electron spectroscopy studies on magneto-optical media and plastic substrate interface,'' IEEE Transl. J. Magn. Japan, vol. 2, pp. 740--741, August 1987 [Digests 9th Annual Conf. Magnetics Japan, p. 301, 1982].
% \bibitem{b7} M. Young, The Technical Writer's Handbook. Mill Valley, CA: University Science, 1989.
\end{thebibliography}
\vspace{12pt}


\end{document}
