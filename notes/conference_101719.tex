\documentclass[conference]{IEEEtran}
\IEEEoverridecommandlockouts
% The preceding line is only needed to identify funding in the first footnote. If that is unneeded, please comment it out.
\usepackage{cite}
\usepackage{amsmath,amssymb,amsfonts}
\usepackage{algorithmic}
\usepackage{graphicx}
\usepackage{textcomp}
\usepackage{xcolor}
\usepackage{parskip}
\def\BibTeX{{\rm B\kern-.05em{\sc i\kern-.025em b}\kern-.08em
    T\kern-.1667em\lower.7ex\hbox{E}\kern-.125emX}}
\begin{document}

\title{Federated Learning Literature Review}


\author{
\IEEEauthorblockN{Seán Ó Fithcheallaigh}
\IEEEauthorblockA{\textit{Department of Computing} \\
\textit{Ulster University}\\
Belfast, N. Ireland \\
sofithcheallaigh@gmail.com or o\_fithcheallaigh-s@ulster.ac.uk}
}

\maketitle

\begin{abstract}
This paper presents a literature review on Federated Learning (FL).

\noindent These notes are for personal use.
\end{abstract}

\begin{IEEEkeywords}

\end{IEEEkeywords}

\section{Introduction}
Federated learning is a relatively new topic in the field of artificial intelligence, and it is closely connected with the emerging field of embedded machine learning, also known as TinyML, or EmbeddedML. Embedded machine learning has become an essential part of the technology domain given the ubiquitous nature of embedded systems and the IoT, and the emerging potential to carry out machine learning tasks at the edge (i.e., in the IoT system or device itself), thereby shifting the implementation of models from high-end systems to low-end clients \cite{b1}.

Federated learning (FL) is a topic that entails the training statistical models for remote devices or data centres. A good example of a remote device is a mobile phone - there are a wide range of mobile phone hardware configurations available and they cover a massive network. From this simple example, we can see that training remote devices in a FL setting means that the heterogeneous nature of the hardware as well as a potentially wide geographical area are just some of the challenges which must be overcome, and will require a radical change from the traditional approach to implementing machine learning models \cite{b2}.

It behoves us to consider, at the outset, some more challenges which will need to be overcome when considering a FL system. \textbf{Data privacy and data security} are always a paramount concern in any system. The architecture of a FL system means that the collected by the remote device will stay on that device, which reduces the risk of data breaches, however, the concerns of data privacy are not completely removed, and as such there is a need to implement schemes to further ensure any data is secure. Since FL involves frequent communication between the central server and the remote devices, another concern is \textbf{communication and bandwidth}. Communication overhead can be a substantial part of the battery life for a resource constrained device. So the need to optimise the communication protocols are is critical to the implementation of a useful FL network. As mentioned above, \textbf{heterogeneous hardware} will require the development of models that can allow for this heterogeneous nature is very important. Depending on the nature of the system, \textbf{data heterogeneity}, where different end devices are collecting different types of data, has the potential to cause problems for the models being implemented, so it is critical to develop a model that will not be hindered by the various types of data, and will allow convergence. Another large challenge is \textbf{model aggregation and model updates}, which involves combining updates from multiple devices. This will require a strategy that will maintain the model's performance and accuracy. Creating this balance between local, on-device learning, and global learning will be extremely important. Associated with this is the need for designing a plan for \textbf{strategy for device participation}. By this we mean, designing a plan to select which devices will take part in each round of training, while preventing others, or preventing devices from dominating the training process. Another potential problem could be \textbf{model overfitting}. This can be a problem because due to the very local nature of the data, there is the potential of overfitting to the noise in that data. Another area for concern is \textbf{convergence and staleness}, which can be an issue due to the fact that different devices will update the model at different times (this is staleness) - ensuring convergence with dealing with these delays is complex. Given the complexities of dealing with wide range of devices, which are updating at different times etc., robustness and fault tolerance will need to be integrated system wide. The system will need to be be able to deal with device failures, dropouts and communications errors without having a significant affect on the over all training process. There are also other concerns around \textbf{regulatory and compliance issues}, such as making sure any usage of data complies with the laws of the region in which the system is being used.


% \section*{References}

\begin{thebibliography}{00}
\bibitem{b1} P. Pratim Ray, "A review on TinyML: State-of-the-art and prospects," in Journal of King Saud University –
Computer and Information Sciences, vol. 34, pp. 1595-1623, 2022, doi: 10.1016/j.jksuci.2021.11.019.

\bibitem{b2} T. Li, A. K. Sahu, A. Talwalkar and V. Smith, "Federated Learning: Challenges, Methods, and Future Directions," in IEEE Signal Processing Magazine, vol. 37, no. 3, pp. 50-60, May 2020, doi: 10.1109/MSP.2020.2975749.
% \bibitem{b2} J. Clerk Maxwell, A Treatise on Electricity and Magnetism, 3rd ed., vol. 2. Oxford: Clarendon, 1892, pp.68--73.
% \bibitem{b3} I. S. Jacobs and C. P. Bean, ``Fine particles, thin films and exchange anisotropy,'' in Magnetism, vol. III, G. T. Rado and H. Suhl, Eds. New York: Academic, 1963, pp. 271--350.
% \bibitem{b4} K. Elissa, ``Title of paper if known,'' unpublished.
 %\bibitem{b5} R. Nicole, ``Title of paper with only first word capitalized,'' J. Name Stand. Abbrev., in press.
 %\bibitem{b6} Y. Yorozu, M. Hirano, K. Oka, and Y. Tagawa, ``Electron spectroscopy studies on magneto-optical media and plastic substrate interface,'' IEEE Transl. J. Magn. Japan, vol. 2, pp. 740--741, August 1987 [Digests 9th Annual Conf. Magnetics Japan, p. 301, 1982].
% \bibitem{b7} M. Young, The Technical Writer's Handbook. Mill Valley, CA: University Science, 1989.
\end{thebibliography}
\vspace{12pt}


\end{document}
